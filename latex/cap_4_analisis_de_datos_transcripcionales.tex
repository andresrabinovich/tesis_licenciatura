\chapter{Análisis de dataset transcripcional Wiegel}
\section{Descripción del dataset}
\section{Métricas transcripcionales}
\section{Clustering}
\subsection{Proceso de filtrado y estandarización de datos}
\subsection{Clustering con k-means}
\subsection{Clustering con dynamic tree cut}
\subsection{Análisis de los métodos y problemas de escala de resolución}
\section{Coherencia entre la métrica transcripcional y otros espacios de conocimiento}
\hl{idea esperamos que los conocimientos (entendidos como nociones de similitud) de los distintos espacios sean diferentes pero no ortogonales...cuantificacion...veamos que estructuras son en cierto grado coherentes}
\subsection{Interacting densities}
\hl{genex1 /genex4  VS BPa/BPb/CC}
\hl{PINinfomap / KEGGinfomap/LCI para referencia}
\subsection{Test de fisher}
\hl{genex1 /genex4  VS BPa/BPb/CC}
\subsection{KTA y zKTA}
\hl{Global}
\hl{KTA Genex por tratamiento + PIN + KEGG + LCI / GOBPa, GOBPb, GOCC}
\hl{zKTA: por tratamiento Gx/GOBPa, Gx/GOBPb, Gx/GOCC, Gx/PIN, Gx/LCI, Gx/Kegg}