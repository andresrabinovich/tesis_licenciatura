\chapter*{Resumen}
Las tecnologías de relevamiento transcripcional a escala global (micromatrices de ADN y RNAseq) se constituyeron como  una herramienta indispensable para el monitoreo de niveles de expresión a lo largo de todo el genoma de un organismo. La enorme cantidad de datos que estas técnicas producen requiere de métodos automatizados para su análisis e interpretación.\\\\
En este trabajo utilizamos el método de agrupamiento de corte de árbol dinámico utilizando como criterio de similaridad entre genes la correlación lineal, para obtener grupos de genes con alta coherencia media (del orden de $0.9$) y caracterizamos las particiones obtenidas para dos resoluciones distintas del método, corte de árbol dinámico con $deepsplit=1$ y corte de árbol dinámico con $deepsplit=4$.\\\\
Luego, utilizamos la información contenida en el espacio de conocimiento biológico GO para cuantificar la congruencia biológica de las particiones halladas. Al obtener bajos niveles de congruencia, buscamos cuantificar la similaridad entre los espacios transcripcional y GO a través de sus respectivas métricas, mediante el alineamiento de núcleo-objetivo (KTA), encontrando una coherencia global no trivial entre la métrica de expresión transcripcional y la métrica del espacio GO.\\\\
Con esta caracterización, incorporamos el lenguaje de redes para definir vecindades transcripcionales a partir de una red de 30 primeros vecinos mutuos y definir un alineamiento de núcleo-objetivo de forma local en estas vecindades. Esto nos permitió desarrollar una métrica mixta a partir de la congruencia biológica de grupos de genes en una vecindad transcripcional, penalizando aquellas relaciones no soportadas por GO e incentivando aquellas que si eran soportadas por GO.\\\\
A partir de esta métrica mixta desarrollamos un método heurístico de agrupamiento para encontrar subestructura en los grupos originales y obtener así subgrupos de genes con alta coherencia media en el espacio transcripcional y con elevada congruencia biológica en el espacio de conocimiento GO.\\\\
Finalmente, buscamos explicar el contenido biológico de los grupos y subgrupos a través de una prueba de sobrerepresentación por medio del test exacto de Fisher. Obtuvimos que los subgrupos de tamaño intermedio, de entre 50 y 60 genes, eran los más explicativos, mientras que los más pequeños, de entre 10 y 20 genes, de alta cohesión y elevada congruencia biológica, eran tan pequeños que no se encontraban sobrerepresentados en ninguna categoría significativa de GO. 