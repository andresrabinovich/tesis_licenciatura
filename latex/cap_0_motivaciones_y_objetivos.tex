\chapter*{Motivaciones y objetivos}
La genómica funcional es un campo de la biología molecular que hace extenso uso de datos genómicos y transcriptómicos para estudiar, describir y responder preguntas acerca de la expresión, función e interacción de genes y proteínas en una escala global (a lo largo de todo el genoma), en contraposición con los métodos más tradicionales de estudio que se realizan gen por gen.\\
Desde principios del año 2000, a partir de la aparición de tecnologías experimentales modernas, tales como la tecnología de Micromatrices de ADN (DNA microarray), secuenciadores de nueva generación (NGS) o secuenciadores de ARN (RNAseq), es posible relevar el estado transcripcional de una célula de forma global, es decir, cuantificar los niveles de todo el RNA mensajero que está siendo exportado en un dado momento desde el núcleo celular hacia el citoplasma con el fin de producir determinadas proteínas.\\
La realización de este tipo de estudios posee un potencial enorme, con aplicaciones tanto en áreas de investigación básica como aplicada, investigaciones biomédicas, farmacológicas y de la salud.\\
En particular, en relevamientos transcripcionales de gran escala es posible obtener información sobre el nivel de activación de miles de genes, para decenas o cientos de condiciones ambientales/experimentales diferentes. Para ganar conocimiento biológico a partir de la cantidad enorme de datos que estos relevamientos generan, es necesario implementar estrategias de búsqueda de correlaciones en espacios de alta dimensionalidad.\\
Para ello, es de fundamental importancia el estudio e implementación de procedimientos de búsqueda de estructuras aplicables a este tipo de relevamientos, cobrando predominancia técnicas estadísticas y técnicas de aprendizaje automático no supervisado, tales como las técnicas de agrupamiento o ``clustering'', que permitan reconocer subconjuntos de genes que evidencien patrones de coexpresión similares a lo largo de conjuntos específicos de condiciones experimentales.\cite{functional_genomics_definition_nature, functional_genomics_definition_wikipedia}

\section*{Objetivos y organización de la tesis}
El presente trabajo tiene como objetivo analizar la coherencia entre la métrica transcripcional y la inferida a partir de otros espacios de conocimiento, como ser redes de interacción de proteinas (PIN por sus siglas en ingles), redes inferidas de literatura curada (LCI), vías metabólicas (KEGG) y ontología genética (GO).\\
Vamos a hacerlo cuantitativamente y tratar de incorporar lo encontrado en la elaboración de métricas mixtas que permitan agrupar perfiles de expresión y obtener estructuras compactas (coherentes) en varios espacios. Para esto utilizaremos el conjunto de datos Wiegel, un exhaustivo estudio de expresión del transcriptoma de Arabidopsis thaliana, dos PINs, AI1 y LCI de \cite{Hahn2013}, una red KEGG de \cite{Kanehisa2000} y una base de datos de anotaciones GO de \cite{org.At.tair.db2015}.\\\\

Está tesis está organizada de la siguiente forma. En el capítulo 1 se introducirán los conceptos biológicos necesarios para comprender y motivar los datos presentados y analizados. En el capítulo 2 introduciremos los materiales y métodos utilizados a lo largo del trabajo. Describiremos la composición y funcionamiento de los métodos de obtención de los cuatro tipos de datos que analizaremos (micromatrices de ADN, redes de interacción de proteínas, redes de vías metabólicas y ontología GO). En el capítulo 3 presentaremos en detalle los métodos de agrupamiento de datos utilizados en este trabajo y analizaremos las problemáticas asociadas a cada uno. En el capítulo 4 analizaremos los datos mediante los métodos presentados en el capítulo 3 y los caracterizaremos buscando información biológica en los mismos. En el capítulo 5 utilizaremos la información obtenida en el capítulo 4 para proponer una métrica mixta que permita aumentar la cantidad de información biológica conseguida previamente. Finalmente, en el último capítulo analizaremos los resultados obtenidos y plantearemos futuras lineas de estudio.