\chapter{Análisis de conjunto de datos transcripcionales Wiegel}
En este capítulo analizaremos el conjunto de datos transcripcionales Wiegel \& Lohmann para la planta Arabidopsis thaliana presentados en la sección \ref{sec:wiegel}, utilizando para ello los métodos de agrupamiento k-means (sección \ref{sec:agrupamientos_no_jerarquicos}) y corte de árbol dinámico híbrido (sección \ref{sec:grupos_en_agrupamiento_jerarquico}) introducidos en el capítulo \ref{materiales_y_metodos} para obtener grupos en el espacio de expresión.\\
Una vez obtenidos los grupos en el espacio de expresión, utilizaremos los índices BHI e Interacting Densities para cuantificar el grado de coherencia entre estas estructuras y los conocimientos (entendidos como nociones de similitud) en el espacio GO.\\
Luego, analizaremos la coherencia de los resultados obtenidos en el espacio de expresión con la de resultados obtenidos en otros espacios de conocimiento, como GO (sección \ref{sec:go}), PIN  (sección \ref{sec:redes}) y KEGG (sección \ref{sec:kegg}), esperando que estos conocimientos sean diferentes pero no ortogonales, utilizando para ello el índice KTA. 
\section{Descripción del dataset}
\hl{esto esta en \label{sec:wiegel}. habra que profundizar mas?}
\section{Métricas transcripcionales}
\hl{esto esta en el capitulo 3, o la idea es poner otra cosa?}
\section{Agrupamiento}

\subsection{Proceso de filtrado y estandarización de datos}
El conjunto de datos Wiegel utilizado consta de los niveles de expresión de 22810 sondas que se mapean a 20149 genes a lo largo de 11 tratamientos diferentes y con entre 4 y 9 muestreos en dos réplicas. Para poder manejar esta cantidad de información es necesario realizar un filtrado previo de los datos que permita quedarse únicamente con aquellos genes que se expresaron o inhibieron, ya que serán estos los genes de interés.\\
Para ello,  
\subsection{Agrupamiento con k-means}

\subsection{Agrupamiento con dynamic tree cut}

\subsection{Análisis de los métodos y problemas de escala de resolución}

\section{Coherencia entre la métrica transcripcional y otros espacios de conocimiento}
\hl{idea esperamos que los conocimientos (entendidos como nociones de similitud) de los distintos espacios sean diferentes pero no ortogonales...cuantificacion...veamos que estructuras son en cierto grado coherentes}
\subsection{Interacting densities}
\hl{genex1 /genex4  VS BPa/BPb/CC}
\hl{PINinfomap / KEGGinfomap/LCI para referencia}
\subsection{Test de fisher}
\hl{genex1 /genex4  VS BPa/BPb/CC}
\subsection{KTA y zKTA}
\hl{Global}
\hl{KTA Genex por tratamiento + PIN + KEGG + LCI / GOBPa, GOBPb, GOCC}
\hl{zKTA: por tratamiento Gx/GOBPa, Gx/GOBPb, Gx/GOCC, Gx/PIN, Gx/LCI, Gx/Kegg}