\chapter*{Resumen}
Las tecnologías de relevamiento transcripcional a escala global (micromatrices de ADN y RNAseq) se constituyeron como una herramienta indispensable para el monitoreo de niveles de expresión a lo largo de todo el genoma de un organismo.\\\\
Para ganar conocimiento biológico a partir de la cantidad enorme de datos que estos relevamientos generan, es necesario implementar estrategias de búsqueda de correlaciones en espacios de alta dimensionalidad, cobrando predominancia técnicas estadísticas y técnicas de aprendizaje automático no supervisado, tales como las técnicas de agrupamiento o ``clustering'', que permitan reconocer subconjuntos de genes que evidencien patrones de coexpresión similares a lo largo de conjuntos específicos de condiciones experimentales.\\\\
En este trabajo desarrollamos un método heurístico capaz de encontrar grupos de genes con alta coherencia en sus patrones de expresión y con alta congruencia biológica, a partir de una métrica mixta que incorpora a la información de perfiles de expresión, el conocimiento biológico de Ontología Génica por medio del lenguaje de redes.\\\\
Luego, a través de una prueba de sobrerepresentación, buscamos explicar el contenido biológico de los grupos hallados, encontrando que los grupos de tamaño intermedio, de entre 50 y 60 genes, eran los más explicativos, mientras que los más pequeños, de entre 10 y 20 genes, de alta cohesión y elevada congruencia biológica, eran tan pequeños que no se encontraban sobrerepresentados en ninguna categoría significativa de GO.\\\\
Estos resultados podrían funcionar como punto de partida para inferir funciones biológicas de genes sin anotaciones en GO o de los que se tiene poco conocimiento biológico.

%hicimos especial énfasis en encontrar grupos de genes con alta coherencia en sus patrones de coexpresión y con alta congruencia biológica. En particular, utilizamos el método de agrupamiento de corte de árbol dinámico y la correlación lineal como criterio de similaridad entre genes, para obtener grupos de genes con alta coherencia media.\\

%Luego, utilizamos la información contenida en el espacio de conocimiento biológico GO para cuantificar la congruencia biológica de las particiones halladas, cuantificando además la similaridad entre los espacios transcripcional y GO a través de sus respectivas métricas. Encontramos coherencia global no trivial entre la métrica de expresión transcripcional y la métrica del espacio GO.\\

%Con esta caracterización, incorporamos el lenguaje de redes para definir vecindades transcripcionales a partir de una red de 30 primeros vecinos mutuos y definir un alineamiento de núcleo-objetivo de forma local en estas vecindades. Esto nos permitió desarrollar una métrica mixta a partir de la congruencia biológica de grupos de genes en una vecindad transcripcional, penalizando aquellas relaciones no soportadas por GO e incentivando aquellas que si eran soportadas por GO.\\

%A partir de esta métrica mixta desarrollamos un método heurístico de agrupamiento para encontrar subestructura en los grupos originales y obtener así subgrupos de genes con alta coherencia media en el espacio transcripcional y con elevada congruencia biológica en el espacio de conocimiento GO.\\

%Finalmente, buscamos explicar el contenido biológico de los grupos y subgrupos a través de una prueba de sobrerepresentación por medio del test exacto de Fisher. Obtuvimos que los subgrupos de tamaño intermedio, de entre 50 y 60 genes, eran los más explicativos, mientras que los más pequeños, de entre 10 y 20 genes, de alta cohesión y elevada congruencia biológica, eran tan pequeños que no se encontraban sobrerepresentados en ninguna categoría significativa de GO. 