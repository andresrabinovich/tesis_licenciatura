
% Este documento LaTeX fue diseñado por profesores  del Departamento de Matemáticas 
% de la Universidad de Antioqua (http://ciencias.udea.edu.co/). Usted puede modificarlo
% y personalizarlo a su gusto bajo los términos de la licencia de documentación libre GNU.
% http://es.wikipedia.org/w/index.php?title=Licencia_de_documentaci%C3%B3n_libre_de_GNU&oldid=15717448

\documentclass[serif,9pt, t]{beamer}
\setbeamertemplate{navigation symbols}{}
\addtobeamertemplate{navigation symbols}{}{
    \insertframenumber/\inserttotalframenumber
}
\setbeamercolor{navigation symbols}{fg=black}
\usetheme{Warsaw}

\usepackage[utf8]{inputenc}
\usepackage[spanish]{babel}
\usepackage{verbatim} %para comentarios multilinea
\usepackage{subfig}

\graphicspath{{../latex/figuras/}}

\newif\ifplacelogo % Para que el logo solo figure en el primer slide
\placelogotrue
\logo{\ifplacelogo\includegraphics[scale=0.25]{logo_exactas}\fi}

\beamersetuncovermixins{\opaqueness<1>{25}}{\opaqueness<2->{15}}

\begin{document}
\title[Análisis y Detección de Correlaciones en Relevamien\ldots ]{Análisis y Detección de Correlaciones en Relevamientos Transcripcionales de Gran Escala}  
\author[Andrés Rabinovich]{Andrés Rabinovich\\{\small Director: Ariel Chernomoretz}}

\institute[Departamento de Física]{
	Departamento de Física\\	
	Facultad de Ciencias Exactas y Naturales\\
	Universidad de Buenos Aires}
\date{Marzo 2016.}


\begin{frame}
\titlepage
\end{frame}

\placelogofalse

\begin{frame}
\frametitle{Contenido}
\tableofcontents
\end{frame} 

\section{Introducción} 

\subsection{Detección de correlaciones}
\begin{frame}\frametitle{Detección de correlaciones} 
\Large Agrupamiento de datos\medskip
\normalsize
\begin{itemize}
\item Método de clasificación no supervisado.
\item Consiste en agrupar elementos ``similares entre si''.
\item Permite el descubrimiento de patrones en los datos.
\item Posibilita obtener conclusiones sobre los datos.
\end{itemize}

\end{frame}

\subsubsection*{Medidas de similaridad}
\begin{frame}
\frametitle{Medidas de similaridad} 
\end{frame}

\subsubsection*{Tipos de agrupamiento}
\begin{frame}
\frametitle{Tipos de agrupamiento} 
\end{frame}

\subsection{Relevamientos transcripcionales de gran escala}

\subsubsection*{Transcripción y traducción}
\begin{frame}
\frametitle{Transcripción y traducción}

\begin{figure}[t!]
  \centering
  \subfloat[Célula eucariota]{\includegraphics[width=0.5\textwidth]{celula_eucariota}}
  \subfloat[Dogma central de la biología molecular]{\includegraphics[width=0.5\textwidth]{adn3}}
\end{figure}

\end{frame}

\subsubsection*{Planta Arabidopsis thaliana (``el oscilador armónico de la biología'')}
\begin{frame}
\frametitle{Cambios transcripcionales en respuesta a estrés abiótico en A.thaliana} 
\end{frame}

\section{Análisis de relevamientos transcripcionales}

\subsection{Métodos km, ds1, ds4}
\begin{frame}
\frametitle{Métodos km, ds1, ds4} 
\end{frame}

\subsection{Caracterización de particiones}
\begin{frame}
\frametitle{Caracterización de particiones} 
\end{frame}

\subsection{El problema de la escala}
\begin{frame}
\frametitle{El problema de la escala} 
\end{frame}

\section{Congruencia biológica}

\subsection{Ontología génica (GO)}
\begin{frame}
\frametitle{Ontología génica (GO)} 
\end{frame}

\subsection{Densidades de interacción}
\begin{frame}
\frametitle{Densidades de interacción} 
\end{frame}

\subsection{Indice de homogeneidad biológica}
\begin{frame}
\frametitle{Indice de homogeneidad biológica} 
\end{frame}

\section{Coherencia entre métricas}
\begin{frame}
\frametitle{Coherencia entre métrica transcripcional y espacio GO} 
\end{frame}

\subsection{KTA global}
\begin{frame}
\frametitle{KTA global} 
\end{frame}

\subsection{Modulación de heterogeneidades
transcripcionales}
\begin{frame}
\frametitle{KTA local para modulación de heterogeneidades transcripcionales} 
\end{frame}
\subsubsection*{Métrica mixta}
\begin{frame}
\frametitle{Métrica mixta} 
\end{frame}
\subsubsection*{Método heurístico}
\begin{frame}
\frametitle{Método heurístico} 
\end{frame}
\subsubsection*{Interpretación biológica}
\begin{frame}
\frametitle{Interpretación biológica} 
\end{frame}
\section{Conclusiones y perspectivas}
\begin{frame}
\frametitle{Conclusiones y perspectivas} 
\end{frame}
\end{document}