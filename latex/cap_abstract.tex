\chapter*{Abstract}
Microarray and RNAseq technology have become one of the indispensable tools that many biologists use to monitor genome wide expression levels of genes in a given organism.\\

In order to gain biological insight from the huge amounts of data that this high-throughput experiments produce, strategies must be developed to search in high-dimensional spaces, paying special attention to unsupervised clustering tecniques that recognice subsets of genes with similar coexpresion patterns through specific sets of conditions.\\

We have developed an heuristic method capable of finding clusters that have both higly coherent expresion patterns and high biological congruence, using a mixed metric that incorporates the biological knowledge from the Gene Ontology to the information contained in expresion profiles through the use of network language.\\

Then, using an overrepresentation test, we tried to explain the biological content of the groups, finding that the middle sized groups, between 50 to 60 genes, were the most informative ones, while the higly coherent smaller groups, between 10 to 20 genes, were so small that no significant GO category could be represented by them.\\

This results could be used as a starting point in order to infer biological functions for genes that aren't annotated in GO or whose biological functions are still unknown.