\chapter{Conclusiones y perspectivas}
Las tecnologías de relevamiento transcripcional a escala global (micromatrices de ADN y RNAseq) se constituyeron como  una herramienta indispensable para el monitoreo de niveles de expresión a lo largo de todo el genoma de un organismo. Las misma permiten estudiar los perfiles de expresión transcripcionales de miles de genes simultáneamente, estimando la concentración de ARNm que está siendo exportado desde el núcleo celular hacia el citoplasma para la síntesis de determinadas proteínas.\\\\
Mediante técnicas de agrupamiento no supervisado de datos es posible encontrar grupos de genes con perfiles de expresión correlacionados, y en particular, utilizando la distancia de correlación lineal considerada en \ref{eq:ccp}, es posible buscar estructuras linealmente correlacionadas. La importancia desde el punto de vista biológico de este tipo de análisis radica en que grupos de genes correlacionados son un primer indicador de que dichos genes podrían formar parte de un mismo proceso biológico.\\\\
Distintos métodos de agrupamiento darán como resultado particiones diferentes en función de la resolución o granularidad que puede alcanzar cada método. Encontrar la resolución óptima es un problema que toma sentido únicamente en referencia a un corpus de datos externo y complementario a una definición matematicamente cerrada del concepto de optimalidad. En nuestro caso resulta relevante poder encontrar agrupamientos de genes que muestren cierto grado de coherencia biológica que permitan elaborar hipótesis razonables. En esa linea, caracterizamos dos métodos de agrupamiento, el método k-means y el método corte de árbol dinámico. De todas formas, encontrar la resolución óptima, excede el alcance de este trabajo.\\\\
El método k-means aplicado a los datos de cambios transcripcionales en respuesta a estrés abiótico en \textit{A. thaliana} presenta una muy baja resolución obteniendo únicamente grupos del orden de 1000 genes, con una coherencia media del orden de $0.7$. Estos grupos tan grandes son de difícil interpretación biológica.\\\\
El método corte de árbol dinámico, por otro lado, es de especial interés por la granularidad que logra alcanzar en el análisis de datos de expresión génica. Es posible regular la resolución del método a partir de variar el parámetro $deepsplit$, con un rango entre $deepsplit=1$ ($ds1$) para particiones más gruesas, y $deepsplit=4$ ($ds4$) para particiones más finas. Encontramos grupos de entre $20$ y $450$ genes para $ds1$ y grupos entre $20$ y $150$ genes para $ds4$, con coherencias medias elevadas en todos los casos de entre $0.85$ y $0.97$. Esta coherencia tan elevada da cuenta de subestructura dentro de los grupos transcripcionales a la que el método k-means no logra acceder.\\\\ 
Para poder dar una interpretación biológica a las particiones obtenidas en el espacio de expresión transcripcional nos valimos de la información contenida en la ontología génica GO.\\\\
Utilizamos el índice densidad de información (ID) para cuantificar el grado en que los genes de una partición comparten anotaciones en GO y además forman parte del mismo grupo en el espacio transcripcional. Medimos además este índice para tres redes distintas, una red de vías metabólicas (KEGG) y dos redes de interacción de proteínas (AI1 y LCI), junto con el índice para un control nulo. Este índice nos permitió verificar que es posible inferir información biológica tanto de las particiones $ds1$ como las $ds4$, siendo las de $ds1$ las que obtuvieron mejor ID entre las dos. Fue entonces de particular interés para nosotros caracterizar las particiones obtenidas con $ds1$.\\\\
Para ello, cuantificamos la homogeneidad biológica de las particiones $ds1$ por medio del índice de homogeneidad biológica (BHI), que reporta, para cada grupo, la máxima proporción de pares de genes agrupados que comparten una misma clase funcional de Ontología Génica. El análisis de BHI indicó que las particiones halladas no pueden ser fácilmente interpretadas a la luz del conocimiento biológico almacenado en GO.\\\\
Buscamos entonces cuantificar la coherencia entre los espacios de expresión genética y de conocimiento biológico desde una perspectiva diferente: desde la métrica en lugar de desde las agrupaciones. Para ello, hicimos uso del alineamiento de núcleo-objetivo (KTA), que da una medida en que dos métricas están alineadas globalmente. Utilizamos la similaridad de correlación definida en \ref{eq:similaridad_de_correlacion} como medida de similaridad para el espacio de expresión y la similaridad semántica entre dos genes \textit{rcmax} definida en \ref{eq:sim_rcmax} para el espacio GO. Encontramos que para todos los tratamientos el índice KTA era superior al del control nulo, indicando una coherencia global no trivial entre la métrica de expresión transcripcional y la métrica del espacio GO.\\\\
Conociendo esto, buscamos entonces la forma de definir agrupamientos de alta coherencia no solo en el espacio transcripcional sino también en el espacio de conocimiento biológico. Para ello, definimos vecindades transcripcionales a partir de una red de 30 primeros vecinos mutuos y con ello una métrica mixta a partir de la congruencia biológica de grupos de genes en una vecindad transcripcional (por medio del índice KTA local).\\\\ Encontramos que el comportamiento del índice KTA local depende linealmente de $wyn_{anotados}$, lo que da cuenta que la información contenida en la ontología GO está relacionada con este índice.\\\\
Para obtener una métrica que permita detectar heterogeneidades dentro de los grupos, modificamos la similaridad de correlación a partir del \textit{stress}, definido en \ref{eq:stress} y un parámetro $\beta$, lo que permite penalizar o incentivar una correlación en el espacio de expresión dependiendo de su similaridad en el espacio GO. A partir de esta métrica desarrollamos tres métodos heurísticos, lkta.dtc, lkta.infomap y lkta.cnm, para buscar estructura dentro de los grupo obtenidos por medio de $ds1$. Además, realizamos un control, llamado \textit{insideX}, utilizando el método $ds4$ sobre los grupos de $ds1$. Obtuvimos para todos los casos de lkta una mejora sustancial en el índice BHI de cada partición, sobre todo con lkta.infomap. Con los nuevos grupos, realizamos además una prueba de sobrerepresentación por medio de un test exacto de Fisher, y encontramos que todos los métodos obtenían subgrupos que podían explicar el contenido biológico del grupo original por si solos, lo que implica la existencia de heterogeneidades no detectadas por $ds1$ en el agrupamiento inicial. De esta manera, encontramos que en general ambas estimaciones de congruencia e interpretabilidad biológica (BHI y test de Fisher) iban en el mismo sentido.\\\\
En los casos de agrupamientos pequeños aparecían desviaciones entre ambas debido principalmente a que en ese caso valores altos de BHI surgían a partir de coincidencias en categorías GO muy inespecíficas (es decir, de alto numero de anotaciones). Esto era particularmente cierto para el caso de lkta.infomap, donde los subgrupos hallados eran tan pequeños que no se encontraban sobrerepresentados en ninguna categoría significativa de GO. En general, esto da un indicio de cual debería ser la resolución óptima para trabajar con este tipo de datos.\\\\
El método desarrollado permitió encontrar grupos con estructuras con alta coherencia en el espacio transcripcional y a la vez con alta congruencia biológica. Esto podría funcionar como punto de partida para inferir funciones biológicas de genes sin anotaciones en GO o de los que se tiene poco conocimiento biológico.\\\\
Por otro lado, sería interesante incorporar evidencia biológica de otros espacios de conocimiento en la métrica mixta, como por ejemplo la contenida en las vías metábolicas o en las redes de interacción de proteínas. La información contenida en estos espacios es información de alta calidad por lo que su inclusión puede agregar valor biológico a este procedimiento.\\\\
Proponemos también, como un trabajo a futuro, analizar otros métodos de agrupamiento para intentar encontrar la resolución óptima que permita extraer los grupos con mayor contenido biológico para cada tratamiento.

