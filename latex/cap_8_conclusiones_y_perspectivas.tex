\chapter{Conclusiones y perspectivas}
La tecnología de micromatrices de ADN se constituyó como una herramienta indispensable para el monitoreo de niveles de expresión a lo largo de todo el genoma de un organismo. Las misma permiten estudiar los perfiles de expresión transcripcionales de miles de genes simultáneamente, estimando la concentración de ARNm que está siendo exportado desde el núcleo celular hacia el citoplasma para la síntesis de determinadas proteínas.\\
Mediante técnicas de agrupamiento no supervisado de datos es posible encontrar grupos de genes con perfiles de expresión correlacionados, y en particular, utilizando la distancia de correlación lineal considerada en \ref{}, es posible buscar estructuras linealmente correlacionadas. La importancia desde el punto de vista biológico de este tipo de análisis radica en que grupos de genes correlacionados   son un primer indicador de que dichos genes forman parte de un mismo proceso biológico.


Las t´
ecnicas de agrupamiento no supervisado aplicadas sobre este tipo de datos per-
miten encontrar grupos de genes cuyos perfiles de expresi´
on est´
en correlacionados. En particular,
la distancia de correlaci´
on lineal considerada (ec.2.1), compara las variaciones de expresi´
on relati-
vas a la media de cada gen, favoreciendo la b´
usqueda de estructuras correlacionadas linealmente.
Este tipo de an´
alisis es muy relevante desde el punto de vista biol´
ogico porque la detecci´
on de
grupos de genes correlacionados es el primer paso para inferir que dichos grupos participen en procesos biol´
ogicos similares.
La
