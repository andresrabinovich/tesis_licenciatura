\chapter{Materiales y Metodos}\label{materiales_y_metodos}
Las técnicas de relevamientos transcripcionales de gran escala, tales como las micromatrices de ADN y secuenciamientos de ARN (secuenciadores de próxima generación), permiten el monitoreo en paralelo de la totalidad del genoma. En este capítulo daremos una introducción al funcionamiento del primer tipo de tecnologías, que será la que usaremos extensivamente en este trabajo, y a los diferentes conjuntos de datos que utilizaremos.\cite{Bose2016}

\section{Micromatrices de ADN}
La tecnología de micromatrices de ADN se constituyó como una herramienta indispensable para el monitoréo de niveles de expresión a lo largo de todo el genoma de un organismo, estimando la concentración de ARNm que está siendo exportado desde el núcleo celular hacia el citoplasma para la síntesis de determinadas proteínas.\\
Una micromatriz es típicamente un portaobjetos de vidrio u otra superficie sólida a la cual se le adosan de forma ordenada y en lugares específicos (llamados sondas) moléculas de ADN. Un mismo sitio puede contener varios millones de copias de moléculas idénticas de ADN de composición conocida (tanto genómico como hebras cortas de oligo-nucleótidos) que se corresponden de forma unívoca con un gen. Una micromatriz de ADN puede medir en simultaneo los niveles de expresión de hasta 40000 genes distintos.\\
En la actualidad, la aparición de tecnologías más rápidas (y cada vez más económicas) de secuenciamiento, conocidas colectivamente como Secuenciamiento de próxima generación (Next-generation sequencing) y RNA-seq, están comenzando a dejar obsoleta la tecnología de micromatrices. Sin embargo, las mismas siguen siendo una herramienta útil en el estudio de los perfiles de expresión genética.\\
Dependiendo de la tecnología utilizada, las micromatrices pueden ser de canal único o de doble canal.\\
\begin{figure}[h]
    \centering
    \includegraphics[width=0.8\textwidth]{micromatriz}
    \caption{Funcionamiento básico de una micromatriz de ADN \hl{Hacer esta figura nuevamente}}
    \label{fig:micromatriz}
\end{figure}
En las micromatrices de un solo canal, las moléculas de ARNm son extraídas de las células de interés del organismo y mediante diversas técnicas son transcritas inversamente a ADN. Luego, el ADN es transcrito nuevamente a ARNm utilizando ARN marcado con un compuesto fluorescente (biotina). Estas copias marcadas y aumentadas son luego colocadas en la micromatriz, permitiendo que el ARNm se difunda por toda la misma.\\
Cuando el ARNm encuentra una sonda que contiene su copia complementaria, se hibridiza con la misma, es decir se pega con una afinidad mucho mayor con la que se puede pegar a cualquier otra. Al lavarse la solución de ARNm, solo aquellos que se hibridizaron con la copia complementaria se mantienen unidos. Finalmente, se ilumina la micromatriz con luz laser de longitud adecuada y se mide la cantidad de fluorescencia emitida por cada sonda. Esta cantidad está asociada a la cantidad de ácido nucléico que se ligó a una dada sonda y eso a su vez será proporcional a la concentración de ese ARNm particular en el tejido de interés.\\
El resultado de un experimento con micromatrices es una tabla o matriz de expresión de $N_g x N_m$ donde cada fila corresponde a los niveles de expresión de cada gen particular ($N_g \approx 20000$ genes), y cada columna a cada muestra ($N_m \approx 15$ muestras) de tejido tomada.\\
Estos tecnologías plantean entonces el problema de como analizar bastas cantidades de datos para obtener información de interés, como ser:
\begin{enumerate}
	\item La identificación de los genes que forman parte de algún proceso biológico
	\item Agrupar tumores para su clasificación clínica
	\item Proveer evidencia de la función de proteínas cuyo rol en el organismo se desconoce
\end{enumerate}
En este trabajo, analizaremos el conjunto de datos Wiegel, datos obtenidos mediante esta tecnología, que detallaremos a continuación.\cite{Babu2004,Schulze2001,Domany2003}
\section{Conjunto de datos transcripcionales Wiegel}\label{sec:wiegel}
En el marco del proyecto AtGenExpress, (un esfuerzo multinacional desarrollado para descubrir el transcriptoma del organismo modelo multicelular Arabidopsis thaliana), el grupo \textit{Weigel \& Lohmann}, de Alemania, realizó en el año 2004 un exhaustivo estudio de expresión del transcriptoma de Arabidopsis thaliana utilizando las micromatrices de ADN Affymetrix ATH1, con el objetivo de comprender las complejas redes de genes que, se conjetura, controlan la tolerancia de la planta al estrés. Para ello, se sometió a plantas de Arabidopsis, de idéntico genotipo y de idénticas condiciones de crecimiento, a diversos tratamientos de estrés.\\
Los tratamientos de estrés se realizaron de tal forma de excluir efectos circadianos (oscilaciones de las variables biológicas en intervalos regulares de tiempo asociadas con un cambio ambiental rítmico), tomando muestras o de la raíz (root) o del tallo (shoot) en dos réplicas biológicas cada 0 minutos, 30 minutos, 1 hora, 3 horas, 6 horas, 12 horas y 24 horas luego del comienzo del tratamiento. En algunos de los tratamientos, por ejemplo, de luz ultravioleta, las alteraciones transcripcionales producto del estrés fueron tan rápidas que se tomaron además muestras a los 15 minutos del comienzo  del mismo. Las muestras de control se tomaron de plantas no sometidas a ningún tratamiento de estrés de la misma forma que con las plantas tratadas.\\
Además de un tratamiento de control, se realizaron los siguientes tratamientos de estrés:
\subsubsection*{Tratamiento de frío}
Las cajas conteniendo las plantas fueron transferidas a hielo para un rápido enfriamiento y mantenidas a 4\degree C en un cuarto frío hasta la cosecha.
\subsubsection*{Tratamiento de calor}
Las cajas fueron transferidas a una incubadora y sometidas a una temperatura de 38\degree C durante 3 horas antes de la cosecha.
\subsubsection*{Tratamiento osmótico y de sal}
Se removieron las balsas de polipropileno que sostienen a las plantas y se agregaron a la solución acuosa, mannitol y NaCl en na concentración final de 300 mM y 150 mM respectivamente. Luego, se devolvieron las balsas a su lugar hasta la cosecha.
\subsubsection*{Tratamiento de heridas}
Se hirió a las plantas utilizando un elemento punzante consistente en 16 agujas, tres veces por hoja, dejando en promedio entre 3 y 4 agujeros.
\subsubsection*{Tratamiento de sequía}
Las plantas fueron expuestas a una corriente de aire durante 15 minuto, lapso durante el cual perdieron un 10\% de su peso. Luego, se las devolvió a la cámara de cultivo hasta la cosecha.
\subsubsection*{Tratamiento con luz ultravioleta B}
Se irradió a las plantas durante 15 minutos con luz ultravioleta B. Bajo estas condiciones se induce una respuesta de la planta tanto para daño por radiación de onda corta como para radiación ultravioleta.

Además de estos tratamientos, se sometió de forma similar a tratamientos oxidativos, de genotoxicidad y de calor y recuperación.
\hl{en el paper no figura la informacion de los tratamientos de genotoxic, oxidative y heat and recovery}
\cite{AtGenExpress, Kilian2007}
\section{PIN - Redes de interacción de proteínas}
\label{sec:redes}
Las redes son construcciones útiles para esquematizar la organización de las interacciones en distinto tipo de sistemas. Las redes permiten tener una vision global de como estan organizadas dichas interacciones.\\
La mayor parte de las funciones biológicas en una célula es llevada a cabo por proteínas a través de procesos de interacción fisica entre ellas, por ejemplo formando complejos proteicos. Por lo tanto, es de fundamental importancia conocer no solo los niveles de expresión de una dada proteína, sino también, en simultaneo, las interacciones fisicas que la misma podria llevar a cabo con otras proteínas. El registro en forma global de estas interacciones conforma lo que se denomina red de interacción de proteínas o PIN, y si la misma contempla la totalidad de las proteínas de una dada especie, la PIN correspondiente se conoce como interactoma completo.\\
\subsection{PIN AI1 y LCI binaria}
A lo largo de esta tesis se analizaron dos redes de interacción de proteínas con el objetivo de utilizarlas como referencia.\\
La primera, una red experimental de interacciones binarias de alta confianza establecida entre 2700 proteinas \cite{Hahn2013} que reporta 5700 interacciones entre las mismas. Para generar este interactoma, el Consorcio de Mapéo del Interactoma de Arabidopsis utilizó una coleccion de aproximadamente 8000 marcos abiertos de lectura (secuencias de ARN comprendidas entre un codón de inicio de traducción y un codón de terminación) representando alrededor del 30\% de los genes codificantes. Probaron todas las interacciones de a pares con un método conocido como \textit{Sistema de doble híbrido} (Y2H por sus siglas en inglés), consistente en la activación de un gen reportero mediante la acción de un factor de transcripción sobre la secuencia regulatoria. En esta técnica, el factor de transcripción es separado en dos fragmentos, uno que reconoce la secuencia regulatoria y otro que promueve la activación de la transcripción. Estos dos fragmentos son luego conectados cada uno a cada una de las dos proteínas (llamadas carnada y presa) que se desean analizar. Si las dos proteínas son capaces de interactuar fisicamente, el factor de transcripción se reconstituirá y se activará el gen reportero, visualizándose como crecimiento en un medio específico o una reacción con cambio de color.\cite{Bruckner2009}\\
Utilizando los pares obtenidos confeccionaron un conjunto de datos consistente en 5664 interacciones binarias entre 2661 proteínas, llamado Arabidopsis Interactome versión 1 ``main screen'', que llamaremos $AI1_{main}$.\\
La segunda red utilizada fue una red binaria de interacción de proteínas, que llamaremos, $LCI_{binaria}$, obtenida de \cite{Hahn2013}, material suplementario, tabla 4, consistente en aproximadamente 4300 interacciones entre alrededor de 2200 proteínas de \textit{Arabidospis}. La misma fue obtenida mediante curado manual de literatura, es decir, en lugar de realizar ensayos de alto rendimiento en busca de pares de proteínas interactuantes, se realiza una revisión exhaustiva de la literatura existente en busca de interacciones que aparezcan en ensayos de pequeña escala previamente realizados sobre pocas proteínas y motivados por hipótesis previas (hypothesis-driven en inglés), ensayos altamente fiables.\cite{Cusick2009}\\
El solapamiento observado entre ambas se encuentra en el rango esperado dado la cobertura del proteoma que hacen estas redes, como muestra el diagrama de la figura \ref{fig:ai1_lci}.
\begin{figure}[h]
    \centering
    \includegraphics[width=0.8\textwidth]{ai1_lci}
    \caption{(Arriba) Representación de las redes $LCI_{binaria}$ (azul) y $AI1_{main}$ (verde). (Abajo a la izquierda) Los conjuntos de datos son representados por diagramas de Venn cuadrados. El tamaño es proporcional al número de interacciones. (Abajo a la derecha) Los solapamientos observados y esperados de $AI1_{main}$ con $LCI_{binaria}$ (fuente: \cite{Hahn2013}} 
    \label{fig:ai1_lci}
\end{figure}
\section{KEGG - Vías metabólicas}
\label{sec:kegg}
Una vía metabólica es un conjunto de reacciones químicas que suceden dentro de una célula. En una vía, la sustancia quimica inicial, llamada metabolito, es modificada por una serie de reacciones químicas catalizadas por enzimas. En estas reacciones, el producto de una enzima es utilizado como substrato por la siguiente enzima y así hasta alcanzar un producto final, que puede usarse inmediatamente, almacenarse o iniciar una nueva vía metabólica. El metabolismo de una célula consise en una red o vías interconectadas que permiten la síntesis o ruptura de las moléculas, acciones conocidas como anabolismo y catabolismo, respectivamente. Descubrir este tipo de redes es fundamental para obtener una imagen global de la actividad celular (figura \ref{fig:mapa_kegg}).\\
La Enciclopedia de Genes y Genomas de Kyoto (KEGG, por sus siglas en inglés), es una base de datos sobre diversos genomas, vías biológicas, enfermedades, drogas y sustancias químicas.
\begin{figure}[h]
    \centering
    \includegraphics[width=0.8\textwidth]{mapa_kegg}
    \caption{Mapa KEGG de la vía metabólica de Arabidopsis Thaliana  \textit{Synthesis and degradation of ketone bodies}}
    \label{fig:mapa_kegg}
\end{figure}
La misma provee de una base de datos de vías metabólicas que contiene recursos para la representación de procesos celulares tales como el metabolismo, transducción de señales y ciclo celular. La figura \ref{fig:mapa_kegg} muestra un mapa KEGG de la vía metabólica de Arabidopsis Thaliana ``Synthesis and degradation of ketone bodies''. En la misma, se puede observar que a priori, la estructura de una vis metabólica excede el lenguaje de redes. Las interacciones metabólicas suelen involucrar sustratos, productos y enzimas en relaciones que son difícil de capturar utilizando únicamente interacciones binarias. 
En nuestro trabajo utilizamos el abordaje de Gabriele Sales y colaboradores \cite{Graphite2015} para mapear vias metabólicas en redes.\\
Se conformó una red uniendo todas las vías metabólicas presentes en la base de datos y teniendo en cuenta solamente aquellos genes presentes en el conjunto de datos Weigel, obteniéndose una red de 1992 nodos y 23009 arcos.
\cite{Segal2003, Kanehisa2000}
\section{GO - Ontología genética}
\label{sec:go}
Poder comparar y clasificar entidades es un mecanismo fundamental de las ciencias biológicas. El advenimiento de tecnologías de alto rendimiento hace que sea necesario adoptar sistemas de representación del conocimiento que sean objetivos y estandarizados. Esto llevó al desarrollo de diversas ontologías para anotación de genes y de sus productos, y en particular, al desarrollo de la Ontología Génica (Gene ontology, GO por sus siglas en inglés).
El proyecto de Ontología Génica (GO) es un consorcio que intenta mantener un vocabulario y una descripción consistente de conceptos biológicos a lo largo de distintas bases de datos. Esta ontología provee un vocabulario controlado de términos definidos para caracterizar las propiedades de productos génicos (proteínas y secuencias de ARN, por ejemplo).\\
El proyecto GO consta de tres ontologías estructuradas que describen los productos génicos en terminos de sus procesos biológicos asociados (ontología \textit{Biological Process}, BP), de sus componentes celulares (ontología \textit{Cellular Component}, CC) y de sus funciones moleculares (ontología \textit{Molecular Function}, MF).\\
Un termino de un proceso biológico (BP) describe una serie de eventos realizados por uno o varios grupos de eventos moleculares con un comienzo y un fin definidos, por ejemplo, ``proceso celular fisiológico'' o ``transducción de señal''. Un proceso biológico no es equivalente a una vía metabólica ya que no intenta representar la dinámica o dependencias de la misma.\\
Un término de componente celular (CC) describe un componente de una célula que es parte de un objeto mayor, como ser una estructura anatómica (por ejemplo, retículo endoplasmático rígido, núcleo, etc.) o un grupo de productos génicos (por ejemplo, ribosoma, proteasoma, etc.).\\
Finalmente, los términos de función molecular (MF) describen las actividades que ocurren a nivel molecular, por ejemplo, ``actividad catalítica'' o ``actividad de transporte''.\\
Cada una de estas tres ontologias está estructurada como un grafo acíclico dirigido (DAG por sus siglas en inglés).\\
Cada nodo representa un término que describe alguna función. Los términos se unen entre si mediante relaciones direccionales del tipo ``es un'' o ``es parte de'', donde el primero expresa una relación de clase-subclase y el segundo una relación de parte-todo (figura \ref{fig:ejemplo_de_go}). Cuando un producto génico es descrito por un termino GO, se dice que el mismo está anotado en ese término, ya sea de forma directa o a través de herencia, ya que estar anotado en un término implica estar anotado en todos los términos ancestrales, regla conocida como \textit{regla del camino verdadero}.\\
\begin{figure}[h]
    \centering
    \includegraphics[width=0.5\textwidth]{ejemplo_de_go}
    \caption{Subgrafo de la ontología BP de GO mostrando el proceso biológico ``Positive regulation of cellular component organization''.}
    \label{fig:ejemplo_de_go}
\end{figure}
Formalmente, podemos describir estás relaciones de la ontología GO de la siguiente manera:\\
Sea $C=\{c_i / 1\leq i \leq N\}$ un conjunto ordenado finito de conceptos que representan términos GO. Los mismos se relacionan entre si a través de las relaciones antes consignadas, de tal forma que $c_i \rightarrow c_j$ denota que $c_i$ es un/es parte de $c_j$. Basado en esto, es posible definir una relación binaria sobre $C$, denotada por $\preceq$, tal que $c_i \preceq c_j$, es decir $c_j$ es un ancestro de $c_i$ en la jerarquía GO. Notar entonces que si $c_k \preceq c_i$ y $c_i \preceq c_j \Rightarrow c_k \preceq c_j$ (regla del camino verdadero). En cada grafo existe un término raíz de la jerarquía $r$, tal que $c_i \preceq r \forall c_i \in C$.\\
Los conceptos más generales se hallarán más próximos al término raíz, mientras que los más específicos e informativos se alejarán del mismo. La anotación de un gen o producto génico se realiza siempre al nodo mas especifico, pudiendo ser anotado además en varios conceptos biológicos a la vez.\\
Una anotación en GO para un dado producto génico consiste en un término GO junto con una referencia que describe el tipo de trabajo o análisis que se realizó para asociar un gen con un término específico. Cada anotación debe además incluir un código de evidencia que indica la forma en que se justifica la anotación a un término particular, lo que le confiere un grado de fiabilidad.\\En particular, existen dos grupos de anotaciones, aquellas que fueron curadas manualmente y aquellas que fueron inferidas de anotaciones electrónicas (IEA). Este último tipo de anotaciones funcionales se realiza de forma automatizada sin que intermedie un curador e involucran comparaciones por similitud de secuencia o anotaciones transferidas de bases de datos y por lo tanto poseen una baja calidad y una gran cobertura, formando alrededor del 40\% de las anotaciones totales. Además, dentro del grupo de las anotaciones que fueron curadas manualmente, se tienen aquellas que fueron inferidas por medio de experimentos (IDA, IEP, IGI, IMP, IPI), aquellas que fueron inferidas por medio de análisis computacional (IBA, ISS, RCA, ISM). La tabla \ref{tab:tipos_de_anotaciones} muestra una descripción de cada código de anotación.\\
\begin{sidewaystable}
\centering
\begin{tabular}{| p{4cm} | p{1cm} | p{3cm} | p{8cm} |}
\hline
Anotación (inferida de)      & Siglas & Tipo & Descripción \\
\hline
Ensayo directo        & IDA & Experimental & Indica que se llevó a cabo un ensayo directo para determinar la función, proceso o componente indicado por el término GO. \\
\hline
Patrón de expresión   & IEP & Experimental & Cubre casos donde la anotación fue inferida por el tiempo o lugar de expresión de un gen. \\
\hline
Interacción genética  & IGI & Experimental & Incluye cualquier combinación de alteraciones en la secuencia (mutaciones) o la expresión de más de un gen/producto génico.\\
\hline
Fenotipo mutante      & IMP & Experimental & Cubre los casos donde la función, proceso o localización celular de un producto génico es inferido basado en diferencias en la función, proceso o localización celular entre dos alelos diferentes del gen correspondiente.\\
\hline
Interacción física    & IPI & Experimental & Cubre interacciones físicas entre la entidad de interés y otra molécula (como ser proteínas, iones o complejos).\\                                    
\hline
Aspecto biológico de un ancestro & IBA & Análisis comp. & Tipo de evidencia filogenética en donde un aspecto de un descendiente es inferido a través de la caracterización de un aspecto de un gen ancestral.\\                                             
\hline
Secuencia o similaridad estructural & ISS & Análisis comp. & Se utiliza cuando un análisis basado en secuencia forma la base de la anotación y la revisión de la misma se realiza de forma manual.\\                                             
\hline
Análisis computacional revisado & RCA & Análisis comp. & Se utiliza para anotaciones realizadas en base a predicciones de análisis computacional de conjuntos de datos de experimentos de gran escala.\\    
\hline                                         
Modelo de secuencia & ISM & Análisis comp. & Se utiliza cuando la evidencia de algún tipo de modelo estadístico es usada para realizar predicciones sobre un producto génico.\\                                             
\hline
\end{tabular}
\caption{Códigos de evidencia GO}
\label{tab:tipos_de_anotaciones}
\end{sidewaystable}
La figura \ref{fig:codigos_de_evidencia} muestra la fracción que representa cada tipo de anotación para cada ontología.\\
Las cantidad total de anotaciones para BP, MF y CC, sin tener en cuenta aquellas pertenecientes a la categoría IEA, totalizan 2540816, 207087 y 1043851 anotaciones respectivamente.
\begin{figure}[h]
    \centering
    \includegraphics[width=0.8\textwidth]{codigos_de_evidencia}
    \caption{Códigos de evidencia en cada una de las ontologías y la fracción del total que representan.}
    \label{fig:codigos_de_evidencia}
\end{figure}
En particular, en este trabajo se tuvieron en cuenta únicamente las evidencias obtenidas experimentalmente. Para ello, se tomaron dos subconjuntos de anotaciones de la ontología BP, que llamaremos BPA, consistente en las anotaciones IDA, IPI, IGI, IMP, con un total de 512235 anotaciones y BPB, consistente en las anotaciones IDA, IPI, IGI, IMP y IEP, con un total de 573688. Además, se utilizó un subconjunto de la ontología CC, consistente en las anotaciones IDA, IPI, IGI, IMP, con un total de 693991 anotaciones. \cite{Pandey2008, Resnik1995, Bose2016, Pesquita2009, Berenstein2014, Ashburner}
